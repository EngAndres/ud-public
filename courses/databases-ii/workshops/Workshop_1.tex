%% filepath: /path/to/workshop/Workshop_1_KaggleSystemsAnalysis.tex
\documentclass[doc]{apa6}

\usepackage[american]{babel}
\usepackage{enumitem}
\usepackage{csquotes}
\usepackage{lipsum}
\usepackage[style=apa,sortcites=true,sorting=nyt,backend=biber]{biblatex}
\DeclareLanguageMapping{american}{american-apa}
\addbibresource{bibliography.bib}

\usepackage{tikz}
\usetikzlibrary{arrows,automata}

\title{Systems Sciences Introduction \\ Semester 2025-I \\ Workshop No. 1 --- Systems Design}
\shorttitle{Systems Sciences Introduction --- Workshop No. 1}

\author{\texttt{Eng. Carlos Andr\'es Sierra, M.Sc.}}
\affiliation{Computer Engineering \\ Universidad Distrital Francisco Jos\'e de Caldas}

\leftheader{C.A. Sierra}

\keywords{Systems Analysis, Systems Engineering, Kaggle Competitions, Chaos Theory, Sensitivity Analysis}

\authornote{
    Carlos Andr\'es Sierra, Computer Engineer, M.Sc.\ in Computer Engineering, Titular Professor at Universidad Distrital Francisco Jos\'e de Caldas.

    Any comment or concern about this document can be sent to Carlos A. Sierra at: \textit{cavirguezs@udistrital.edu.co}.
}

\begin{document}
\maketitle

Welcome to the first workshop of the \textit{Systems Sciences} course! This workshop 
focuses on \textbf{systems design} for: an \emph{Autonomous Adaptive Agent Simulation}. 
By exploring cybernetic principles, reinforcement learning, and environment-driven decision-making, 
you will lay the groundwork for creating a self-regulating, intelligent system.

~\\

\noindent \textbf{Workshop Scope and Objectives:}
\begin{itemize}[leftmargin=1.4cm]
    \item \textbf{Systems Design Framework:} Understand the project’s requirements and 
            structure a foundational \textit{systems design} for an autonomous agent.
    \item \textbf{Cybernetic Principles:} Identify feedback loops, sensors, and 
            decision-making mechanisms that enable dynamic adaptation in the simulated 
            environment.
    \item \textbf{Reinforcement Learning Path:} Outline how learning based on rewards be 
            incorporated into the design to optimize agent actions.
    \item \textbf{Scalability \& Extension:} Consider multi-agent and collaborative 
            features for future development phases.
\end{itemize}

\newpage
\noindent \textbf{Methodology and Deliverables:}
\begin{enumerate}
    \item \textbf{System Requirements Document:}
    \begin{itemize}
        \item \textit{Functional Specifications:} Detail how sensors, actuators, and 
                reward functions integrate into the environment.
        \item \textit{Use Cases:} Describe agent-environment interactions, including 
                learning objectives and adaptation goals.
    \end{itemize}

    \item \textbf{High-Level Architecture:}
    \begin{itemize}
        \item \textit{Component Diagram:} Show major modules (e.g., sensor module, RL 
            module, environment module) and data flow.
        \item \textit{Feedback Loops:} Illustrate cybernetic control loops for 
            self-regulation within the agent.
    \end{itemize}

    \item \textbf{Preliminary Implementation Outline:}
    \begin{itemize}
        \item Identify potential frameworks (Gymnasium, Stable-Baselines3) and explain 
                why they are suitable for your design.
        \item Sketch a timeline for moving from basic Q-learning to more advanced DQN 
                approaches.
    \end{itemize}

    \item \textbf{GitHub Repository:}
    \begin{itemize}
        \item Create a  \texttt{GitHub} repository for the course with a folder 
                \texttt{Workshop-1} to store your System Requirements Document, 
                Architecture Diagrams, and any auxiliary notes.
        \item Link your design documentation in a \texttt{README.md}, referencing any 
                code snippets and diagrams used.
    \end{itemize}
\end{enumerate}

\noindent \textbf{Deadline:} \textbf{Wednesday, April 9th, 2025, 8:00}. Submissions after 
this deadline may incur penalties in accordance with course policies.

\bigskip
\noindent \textbf{Notes:}
\begin{itemize}
    \item Keep your report in \textbf{English} and submit it as a \textbf{PDF}.
    \item Cite external sources (papers, tutorials, articles) as needed. 
    \item Your submission should emphasize the \textbf{systems design} aspect, 
        preparing the foundation for future workshops, where you will refine, simulate, 
        and ultimately implement your autonomous agent.
\end{itemize}

\noindent \textit{Good luck, and remember: this workshop is your starting point for 
        conceptualizing and designing a self-adaptive, cybernetic agent. Use the 
        principles of \textit{systems design} to ensure your final project is both 
        robust and future-ready}.

\end{document}